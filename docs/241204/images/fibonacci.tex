% example.tex
\documentclass[dvisvgm]{standalone}

\usepackage{amsmath}
\usepackage[usenames,dvipsnames]{xcolor}
\usepackage{amsmath}
\usepackage{tikz}
\usetikzlibrary {angles,
                 arrows.meta,
                 calc,
                 positioning,
                 shapes.geometric}

 \tikzset{
        square/.style={regular polygon, regular polygon sides=4},
        base/.style={draw, align=center, minimum height=4ex},
        proc/.style={base, rectangle, text width=8em},
        io/.style={trapezium, trapezium left angle=70, trapezium right
                   angle=110, draw, text width=8em, %minimum width=2cm, 
                   %minimum height=1cm
                   },
        test/.style={base, diamond, aspect=2,
                     %text width=5em
                     },
        term/.style={proc, rounded corners},
        myarrow/.style={-Stealth, line width=0.25mm},
 }

\begin{document}
\begin{tikzpicture}
    [level 1/.style={sibling distance=40mm},
     level 2/.style={sibling distance=30mm},
     level 3/.style={sibling distance=20mm},
    ]
    \node {$f(5)$}
    child {node {$f(4)$}
         child{node {$f(3)$}
             child{node {$f(2)$}}
             child{node {$f(1)$}}}
         child{node {$f(2)$}}}
    child {node {$f(3)$}
        child{node {$f(2)$}}
        child{node {$f(1)$}}};

\end{tikzpicture}
\end{document}
